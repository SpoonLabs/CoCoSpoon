\chapter*{Conclusion}
	\thispagestyle{conclusion}
	\addcontentsline{toc}{chapter}{Conclusion}
	
\par CoCoSpoon constitue une solution élégante afin de répondre à la question initiale : peut-on connaître en temps réel la couverture de code exécutée en production ? Cependant, nous soulignons que la mise en place de notre outils introduit un overhead non-négligeable sur le temps d'exécution. Il pourrait être utilisé sur des applications en phase bêta pour détecter le code mort. En effet, la monté en charge de ce type d'application est plus faible et les attentes des utilisateurs sont généralement moins exigeantes.
\par Notre instrumentation permet actuellement de mesurer la couverture de ligne. Il pourrait être intéressant d'étendre l'outil pour mesurer d'autres types de couvertures, tels que la couverture de branches ou de méthodes. \par Une nouvelle métrique pourrait être également ajoutée : le nombre de fois qu'une ligne est exécutée. Celle-ci permettrait de cibler les parties intéressantes à optimiser, ces dernières étant utilisées de nombreuses fois. \par On pourrait également imaginer un plugin Maven automatisant le workflow de CoCoSpoon.

~\\
~\\
~\\
~\\

\centering
\includegraphics[scale=0.6]{logoCoCoSpoon.png}
